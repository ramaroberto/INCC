
\documentclass[12pt,journal,compsoc]{IEEEtran}
\usepackage[spanish]{babel}
\usepackage[utf8]{inputenc}
\newcommand\MYhyperrefoptions{bookmarks=true,bookmarksnumbered=true,
pdfpagemode={UseOutlines},plainpages=false,pdfpagelabels=true,
colorlinks=true,linkcolor={black},citecolor={black},urlcolor={black},
pdftitle={Estudio sobre la relación entre la dificultad del vocabulario y la calificación obtenida de libros},
pdfsubject={Neurociencia},
pdfauthor={Gustavo Juantorena, Roberto Rama, Sabrina Izcovich},
pdfkeywords={Computer Society, IEEEtran, journal, LaTeX, paper,
             template}}

\hyphenation{op-tical net-works semi-conduc-tor}
\begin{document}
\title{Estudio sobre la relación entre la calificación y la dificultad del vocabulario en libros\\ para INCC}

\author{Gustavo~Juantorena,~\IEEEmembership{Estudiante~de~Lic.~en~Cs.~Biológicas,~UBA,}
        Roberto~Rama,~\IEEEmembership{Estudiante~de~Lic.~en~Cs.~de~la~Computación,~UBA,}
        y~Sabrina~Izcovich,~\IEEEmembership{Estudiante~de~Lic.~en~Cs.~de~la~Computación,~UBA,}}

\IEEEtitleabstractindextext{
\begin{abstract}
En el siguiente trabajo, se intenta encontrar una relación entre la dificultad de vocabulario encontrada en libros literarios y las calificaciones obtenidas por parte de sus lectores. Para ello, se utilizó una lista de palabras catalogadas como Simple English\footnote{$http://simple.wikipedia.org/wiki/Main\_Page$}, la base de datos de Amazon\footnote{http://snap.stanford.edu/data/amazon-meta.html} y distintas métricas que nos permitieron extraer conclusiones al respecto. %acá sino podríamos decir: que nos permitieron concluir blabla
\end{abstract}


\begin{IEEEkeywords}
Neurociencia, Simple English, Vocabulario, Reseñas.
\end{IEEEkeywords}}

\maketitle
\IEEEdisplaynontitleabstractindextext
\IEEEpeerreviewmaketitle

\section{Introducción}
\IEEEPARstart{E}{n la actualidad,} el avance de la tecnología permite compartir experiencias de distinta índole. Entre ellas, el intercambio de críticas y comentarios sobre diferentes productos pueden causar la popularidad o decadencia del mismo, llevando a un aumento o disminución de sus ventas. A partir de esto, surgen dudas sobre los motivos de los usuarios a la hora de calificar. En este caso, nos pareció interesante analizar las puntuaciones recibidas por los libros respecto del vocabulario empleado en los mismos. 

En el siguiente trabajo, se presenta un estudio exhaustivo que busca encontrar una correlación entre las calificaciones que reciben distintos libros a través de un sitio de Compra/Venta de internet ($Amazon$), y el grado de dificultad del vocabulario encontrado en los mismos. Para ello, fue menester investigar una amplia variedad de libros escritos por distintos autores, de distintos géneros y formatos, y con distintas calificaciones recibidas por sus lectores. Por otro lado, debimos considerar una forma de medir el grado de dificultad del inglés utilizado. Para ellos, tuvimos en cuenta la categorización de Simple English de Wikipedia.

La idea del proyecto consiste en poder predecir si futuros comentarios sin score son buenos o malos, o la sugerencia de productos basada en productos que ya hayamos comprado. Entre estos productos se encuentran los libros. Nuestro analisis se focalizo, a diferencia de las tecnicas mencionadas anteriormente, en relacionar el producto mismo (ya que al ser un libro puede ser analizado) con el score que recibio de los lectores. Nuestra hipotesis es que utilizando alguna o varias metricas puede analizarse el contenido del libro y relacionarse de forma significativa con su puntaje.
Tambien describe como esta relacionado el paper y que es lo que se explica en cada seccion.

\section{Trabajos relacionados:}

Para orientar nuestra investigación revisamos documentos del área y encontramos varios estudios realizados. En uno de los más relevantes (\textit{Graesser, A. C. et al., 2004}), los autores utilizaron un software para clasificar los textos según la dificultad de lectura que presentan (avanzado, intermedio, principiante). De esta forma, se intentó probar si existen diferencias significativas entre dos métodos para evaluar comprensibilidad de textos simplificados en inglés.\\
Por otro lado, a partir de otro trabajo del mismo grupo (\textit{Crossley, S. A., et al., 2011}), pudimos orientarnos en lo que respecta el procesamiento automático de textos con el fin de obtener algún parámetro numérico sobre su contenido. En el mismo, los autores desarrollaron una herramienta computacional para medir coherencia, legibilidad y otros constructos en textos de forma sencilla en un gran corpus de texto.\\
Finalmente, el trabajo presentado durante las clases teóricas (\textit{Diuk, C. G., et. al., 2012}), resultó de gran utilidad como inspiración sobre la potencialidad del análisis de grandes repositorios de datos con el fin de hacer predicciones. En el mismo, se utilizaron  textos de distintas épocas en orden cronológico con el fin de probar si el constructo introspección creció o disminuyó a lo largo de los años.

\section{Explicación:} 
explicar el programa que hicimos medio por arriba
hay que decir que para ver que funcionara bien nuestro programa lo probamos con los libros de harry potter y vimos que el nivel de dificultad del vocabulario aumenta con los libros, lo que es consistente respecto de lo que dice la autora
Antes de entrar a las subsecciones deberiamos explicar por arriba lo que realizamos. Estaria bueno dar la idea de lo que queremos hacer: ``Para nuestro analisis tuvimos que hayar una base de datos de alguna tienda online donde se detalle cada producto y el score que recibio por parte de los usuarios. Seleccionamos libros de las categorias mas generales con scores opuestos y analizamos las mediciones con respecto a una metrica creada.''
\subsection{Recursos utilizados:}
 Explicar de donde sacamos la base de datos de amazon y detallar su composicion. Aca deberiamos decir tambien los problemas que encontramos con la misma, como las categorias y que habia productos repetidos. Podriamos mostrar un histograma en donde se vea la distribucion de los puntajes en relacion a la cantidad de libros y otro con la cantidad de scores (para que se vea que hay pocos libros con muchos scores y muchos con pocos). Podemos explicar que la obtencion de los libros en si se dificulto un poco para los puntajes bajos y en donde la cantidad de scores era menor a 100.
\subsection{Filtrado de los libros:} Aca podemos explicar como hicimos para ordenar los libros de la lista para seleccionar a mano y como funciona el script (getBooks.py)
\subsection{Métricas utilizadas:} Explicacion de la metrica de simple english y de la metrica de repeticiones (quizas?). Explicacion del algoritmo utilizado para generar la metrica, explicar por que utilizamos un diccionario para filtrar palabras positivas. Explicar cantidad de apariciones (que al final no nos sirvio) contra cantidad de vocabulario simple (que es la que se presento relevante). Explicar como hicimos para validar la metrica (libros de whinny pooh, textos de harry potter vs edades recomendadas). No dar resultados de las metricas, eso se da en la proxima seccion.

\section{Resultados:}
\subsection{Elección de libros analizados:} Aca podemos justificar el armado de las categorias ``Buenos'' y ``Malos'' y porque decidimos dejar los que estaban calificados como 4.0 fuera. Ademas hay que explicar cuantos libros agarramos de cada categoria principal y porque decidimos dejar los que tenian menos de 100 reviews fuera (basicamente, es por la convergencia del promedio de una muestra). \subsection{Resultados de las métricas:} Podemos mostrar graficos de histogramas combinados, uno para los resultados de los libros ``Buenos'' y otros para los ``Malos'' de las 3 metricas: cantidad de palabras simple english, cantidad de vocabulario simple english y cantidad de repeticiones. Podemos decir que en un principio decidimos eliminar la cantidad de palabras como una metrica ya que da siempre entre 40\% y 50\% para todos los libros.\\

\subsection{Test de hipótesis:} Lo primero que va aca es la justificacion por el teorema central del limite de que podemos considerar los datos provenientes de una distribucion normal. Hay que citar alguna fuente en donde clarifique que arriba de 30 es aceptable utilizarlo. Luego, para cada metrica mostramos los resultados de las medias y la varianza y el resultado del test de hipotesis.

\section{Conclusiones y trabajo futuro:} Por lo que vi vamos a tener que decir que obtuvimos resultados significativos pero los numeros estan demasiado cercanos como para utilizar la metrica como criterio. Explicar que otras metricas se nos ocurren para lo cual podria funcionar. Por ejemplo, a mi se me ocurre que una metrica mutli-variable con elementos de procesamiento de lenguaje natural podria dar mejores resultados, es decir metricas que describan mejor la forma en que el libro esta escrito.\\

%\begin{figure}[!t]
%\centering
%\includegraphics[width=2.5in]{myfigure}
% where an .eps filename suffix will be assumed under latex, 
% and a .pdf suffix will be assumed for pdflatex; or what has been declared
% via \DeclareGraphicsExtensions.
%\caption{Simulation Results.}
%\label{fig_sim}
%\end{figure}

% An example of a double column floating figure using two subfigures.
% (The subfig.sty package must be loaded for this to work.)
% The subfigure \label commands are set within each subfloat command,
% and the \label for the overall figure must come after \caption.
% \hfil is used as a separator to get equal spacing.
% Watch out that the combined width of all the subfigures on a 
% line do not exceed the text width or a line break will occur.
%
%\begin{figure*}[!t]
%\centering
%\subfloat[Case I]{\includegraphics[width=2.5in]{box}%
%\label{fig_first_case}}
%\hfil
%\subfloat[Case II]{\includegraphics[width=2.5in]{box}%
%\label{fig_second_case}}
%\caption{Simulation results.}
%\label{fig_sim}
%\end{figure*}


% An example of a floating table. Note that, for IEEE style tables, the 
% \caption command should come BEFORE the table. Table text will default to
% \footnotesize as IEEE normally uses this smaller font for tables.
% The \label must come after \caption as always.
%
%\begin{table}[!t]
%% increase table row spacing, adjust to taste
%\renewcommand{\arraystretch}{1.3}
% if using array.sty, it might be a good idea to tweak the value of
% \extrarowheight as needed to properly center the text within the cells
%\caption{An Example of a Table}
%\label{table_example}
%\centering
%% Some packages, such as MDW tools, offer better commands for making tables
%% than the plain LaTeX2e tabular which is used here.
%\begin{tabular}{|c||c|}
%\hline
%One & Two\\
%\hline
%Three & Four\\
%\hline
%\end{tabular}
%\end{table}


%\ifCLASSOPTIONcaptionsoff
%  \newpage
%\fi

\begin{thebibliography}{1}

\bibitem{IEEEhowto:kopka}
Graesser,~A.~C.,~McNamara,~D.~S.,~Louwerse,~M.~M., \& Cai,~Z. (2004). ``Coh-Metrix: Analysis of text on cohesion and language.''\hskip 1em plus
  0.5em minus 0.4em\relax \emph{Behavior Research Methods,Instruments, \& Computers}, 36(2), 193-202.

\bibitem{IEEEhowto:kopka}
Crossley,~S.~A.,~Allen,~D.~B., \& McNamara,~D.~S. (2011). ``Text Readability and Intuitive Simplification: A Comparison of Readability Formulas.''\hskip 1em plus
  0.5em minus 0.4em\relax \emph{Reading in a foreign language}, 23(1), 84-101.

\bibitem{IEEEhowto:kopka}
Diuk,~C.~G.,~Slezak,~D.~F.,~Raskovsky,~I.,~Sigman,~M., \& Cecchi,~G.~A. (2012). ``A quantitative philology of introspection.''\hskip 1em plus
  0.5em minus 0.4em\relax \emph{Frontiers in integrative neuroscience}, 6.

\end{thebibliography}

\end{document}


